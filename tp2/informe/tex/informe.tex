
\documentclass[a4paper, 10pt, twoside]{article}

\usepackage[top=1in, bottom=1in, left=1in, right=1in]{geometry}
\usepackage[utf8]{inputenc}
\usepackage[spanish, es-ucroman, es-noquoting]{babel}
\usepackage{setspace}
\usepackage{fancyhdr}
\usepackage{lastpage}
\usepackage{amsmath}
\usepackage{bera}% optional: just to have a nice mono-spaced font
\usepackage{listings}
\usepackage{xcolor}
\usepackage{amsfonts}
\usepackage{amsthm}
\usepackage{verbatim}
\usepackage{fancyvrb}
\usepackage{graphicx}
\usepackage{float}
\usepackage{enumitem} % Provee macro \setlist
\usepackage{tabularx}
\usepackage{multirow}
\usepackage{hyperref}
\usepackage{xspace}
\definecolor{lightgray}{rgb}{.9,.9,.9}
\definecolor{darkgray}{rgb}{.4,.4,.4}
\definecolor{purple}{rgb}{0.65, 0.12, 0.82}
\usepackage{ulem} % Provee macro \uwave
\usepackage[toc, page]{appendix}


\lstdefinelanguage{JavaScript}{
  keywords={typeof, new, true, false, catch, function, return, null, catch, switch, var, if, in, while, do, else, case, break},
  keywordstyle=\color{blue}\bfseries,
  ndkeywords={class, export, boolean, throw, implements, import, this},
  ndkeywordstyle=\color{darkgray}\bfseries,
  identifierstyle=\color{black},
  sensitive=false,
  comment=[l]{//},
  morecomment=[s]{/*}{*/},
  commentstyle=\color{purple}\ttfamily,
  stringstyle=\color{red}\ttfamily,
  morestring=[b]',
  morestring=[b]"
}

\lstset{
   language=JavaScript,
   backgroundcolor=\color{lightgray},
   extendedchars=true,
   basicstyle=\footnotesize\ttfamily,
   showstringspaces=false,
   showspaces=false,
   numbers=left,
   numberstyle=\footnotesize,
   numbersep=9pt,
   tabsize=2,
   breaklines=true,
   showtabs=false,
   captionpos=b
}



%%%%%%%%%% Constantes - Inicio %%%%%%%%%%
\newcommand{\titulo}{Trabajo Práctico 2}
\newcommand{\materia}{Bases de Datos}
\newcommand{\integrantes}{Hosen · Marto Caravario · Rajgnewerc · Russo}
\newcommand{\cuatrimestre}{Primer Cuatrimestre de 2016}
%%%%%%%%%% Constantes - Fin %%%%%%%%%%


%%%%%%%%%% Configuración de Fancyhdr - Inicio %%%%%%%%%%
\pagestyle{fancy}
\thispagestyle{fancy}
\lhead{\titulo · \materia}
\rhead{\integrantes}
\renewcommand{\footrulewidth}{0.4pt}
\cfoot{\thepage /\pageref{LastPage}}

\fancypagestyle{caratula} {
   \fancyhf{}
   \cfoot{\thepage /\pageref{LastPage}}
   \renewcommand{\headrulewidth}{0pt}
   \renewcommand{\footrulewidth}{0pt}
}
%%%%%%%%%% Configuración de Fancyhdr - Fin %%%%%%%%%%


%%%%%%%%%% Miscelánea - Inicio %%%%%%%%%%
% Evita que el documento se estire verticalmente para ocupar el espacio vacío
% en cada página.
\raggedbottom

% Separación entre párrafos.
\setlength{\parskip}{0.5em}

% Separación entre elementos de listas.
\setlist{itemsep=0.5em}

% Asigna la traducción de la palabra 'Appendices'.
\renewcommand{\appendixtocname}{Apéndices}
\renewcommand{\appendixpagename}{Apéndices}
%%%%%%%%%% Miscelánea - Fin %%%%%%%%%%


%%%%%%%%%% Macros para el modelo relacional - Inicio %%%%%%%%%%
\newcommand{\relacion}[3]{
  \noindent
  \textbf{#1}(\ignorespaces#2\unskip) \\
  #3
  \vspace{0.5em}
}
\newcommand{\pk}[1]{%
  \underline{#1}%
}
\newcommand{\fk}[1]{%
  \uwave{#1}%
}
\newcommand{\pkfk}[1]{%
  \pk{\fk{#1}}%
}
\newcommand{\clavespkck}[1]{
  PK = CK = \{#1\}
}
\newcommand{\clavespkckfk}[1]{
  PK = CK = FK = \{#1\}
}
\newcommand{\clavesfk}[1]{
  FK = \{#1\}
}
%%%%%%%%%% Macros para el modelo relacional - Fin %%%%%%%%%%



\colorlet{punct}{red!60!black}
\definecolor{background}{HTML}{EEEEEE}
\definecolor{delim}{RGB}{20,105,176}
\colorlet{numb}{magenta!60!black}

\lstdefinelanguage{json}{
    basicstyle=\normalfont\ttfamily,
    numbers=left,
    numberstyle=\scriptsize,
    stepnumber=1,
    numbersep=8pt,
    showstringspaces=false,
    breaklines=true,
    frame=lines,
    backgroundcolor=\color{background},
    literate=
     *{0}{{{\color{numb}0}}}{1}
      {1}{{{\color{numb}1}}}{1}
      {2}{{{\color{numb}2}}}{1}
      {3}{{{\color{numb}3}}}{1}
      {4}{{{\color{numb}4}}}{1}
      {5}{{{\color{numb}5}}}{1}
      {6}{{{\color{numb}6}}}{1}
      {7}{{{\color{numb}7}}}{1}
      {8}{{{\color{numb}8}}}{1}
      {9}{{{\color{numb}9}}}{1}
      {:}{{{\color{punct}{:}}}}{1}
      {,}{{{\color{punct}{,}}}}{1}
      {\{}{{{\color{delim}{\{}}}}{1}
      {\}}{{{\color{delim}{\}}}}}{1}
      {[}{{{\color{delim}{[}}}}{1}
      {]}{{{\color{delim}{]}}}}{1},
}





\begin{document}


%%%%%%%%%%%%%%%%%%%%%%%%%%%%%%%%%%%%%%%%%%%%%%%%%%%%%%%%%%%%%%%%%%%%%%%%%%%%%%%
%% Carátula                                                                  %%
%%%%%%%%%%%%%%%%%%%%%%%%%%%%%%%%%%%%%%%%%%%%%%%%%%%%%%%%%%%%%%%%%%%%%%%%%%%%%%%


\thispagestyle{caratula}

\begin{center}

\includegraphics[height=2cm]{DC.png} 
\hfill
\includegraphics[height=2cm]{UBA.jpg} 

\vspace{2cm}

Departamento de Computación,\\
Facultad de Ciencias Exactas y Naturales,\\
Universidad de Buenos Aires

\vspace{4cm}

\begin{Huge}
\titulo
\end{Huge}

\vspace{0.5cm}

\begin{Large}
\materia
\end{Large}

\vspace{1cm}

\cuatrimestre

\vspace{4cm}

\begin{tabular}{|c|c|c|}
\hline
Apellido y Nombre & LU & E-mail\\
\hline
Federico Hosen  & 825/12 & federico.hosen@gmail.com\\
Martin 'Marto' Caravario  & 470/12 & martin.caravario@gmail.com\\
Guido Rajngewerc  & 379/12 & guido.raj@gmail.com\\
Christian Russo  & 679/10 & christian.russo8@gmail.com\\
\hline
\end{tabular}

\end{center}

\newpage


%%%%%%%%%%%%%%%%%%%%%%%%%%%%%%%%%%%%%%%%%%%%%%%%%%%%%%%%%%%%%%%%%%%%%%%%%%%%%%%
%% Introducción                                                              %%
%%%%%%%%%%%%%%%%%%%%%%%%%%%%%%%%%%%%%%%%%%%%%%%%%%%%%%%%%%%%%%%%%%%%%%%%%%%%%%%


\section{Introducción}

En el siguiente trabajo se desarrolla un modelo de Bases de Datos NoSQL con MongoDB teniendo el cuenta el problema descripto en el TP1.
Para esto se utilizo una seccion del DER completo del TP1 y se diseñaron los documentos correspondientes para realizar las consultas pedidas en el enunciado del TP2.

%%%%%%%%%%%%%%%%%%%%%%%%%%%%%%%%%%%%%%%%%%%%%%%%%%%%%%%%%%%%%%%%%%%%%%%%%%%%%%%
%% Parte 1                                                                   %%
%%%%%%%%%%%%%%%%%%%%%%%%%%%%%%%%%%%%%%%%%%%%%%%%%%%%%%%%%%%%%%%%%%%%%%%%%%%%%%%
\section{Parte 1 - Diseño}

\subsection{Diagrama Entidad relacion}

\begin{figure}[h]
\includegraphics[width=18cm]{der}
\end{figure}

Las consultas pedidas solamente necesitan de usuarios, ventas, reputaciones, calificaciones, comisiones y facturas. Por lo tanto usamos:

\begin{itemize}
\item Para usuarios y ventas, las tablas de Usuarios, Publicaciones y Compra
\item Para reputaciones de usuarios, las tablas de Calificacion y Usuario
\item Para comisiones, usamos las tablas de Publicacion y Tipo de Publicacion.
\item Para monto de facturas, usamos las tabla de Facturas
\end{itemize}

\newpage
\subsection{Colecciones}

A continuacion mostramos las colecciones que utilizamos para resolver las consultas propuestas en el enunciado

\subsubsection{Publicaciones}

\begin{lstlisting}[language=json,firstnumber=1]
{
  "Id": 65,
  "TipoPublicacion": "Servicio"
}
\end{lstlisting}

\subsubsection{Compra}


\begin{lstlisting}[language=json,firstnumber=1]
{
  "idCompra": 656897,
  "idUsuarioComprador": 65468,
  "Fecha": "2-06-2014",
  "Cantidad": 2,
  "Publicacion": {
    "idPublicacion": 2365,
    "idUsuario": 123,
    "Precio": 3655,
    "PorcentajeVenta": 20 //entre 0 y 100
  },
  "CalificacionDelVendedor": 8,
  "CalificacionDelComprador": 6
}
\end{lstlisting}


\subsubsection{Factura}

\begin{lstlisting}[language=json,firstnumber=1]
{
  "idFactura": 687468,
  "IdUsuario": 654,
  "Fecha": "16-9-2015",
  "TotalAPagar": 5624,
  "estoyPagando": [
    {
      "idPublicacion": 54,
      "TipoSuscripcion": "Rubi"
    },
    {
      "idPublicacion": 54,
      "TipoSuscripcion": "Libre"
     }
  ]
}
\end{lstlisting}
\newpage
\subsection{Migracion datos SQL a Colecciones}
Para migrar los datos de SQL (MySQLWorkwench) al formato de JSON que se necesita en MongoDB lo que hicimos fue exportar los datos directo desde MySQLWorkwench, pero estos se exportan en JSON plano, es decir no quedaba formateado de la forma que nosotros queriamos. 
Para solucionar este problema realizamos unos algoritmos en python para formatearlos a mano. Estos algoritmos reciben como input el JSON plano y devuelven un JSON formateado para MongoDB

\subsubsection{Parser para el JSON de la compra}

\begin{verbatim}
import json
from pprint import pprint

with open(raw_input()) as data_file:    
    data = json.load(data_file)
for a in data:
  print "{"
  print '"idCompra": ' , a['idCompra'], ","
  print '"idUsuario": ', a['idUsuario'],","
  print '"fecha": ', '"',a['fecha'],'"',","
  print '"cantidad": ', a['cantidad'],","
  print '"Publicacion": {'
  print '"idPublicacion": ', a['idPublicacion'],","
  print '"idUsuario": ', a['idUsuario'],","
  print '"Precio": ', a['precio'],","
  print '"PorcentajeVenta": ', a['porcentajeVenta']
  print "}" ,","
  print '"calificacionVendedor": ', a['CalificacionDelVendedor'], ','
  print '"calificacionComprador": ', a['CalificacionDelComprador'], ''
  print '}'

\end{verbatim}

\newpage
\subsubsection{Parser para el JSON de la Facutra}

\begin{verbatim}
import json
from pprint import pprint

with open('factura_sql.json') as data_file:    
    data = json.load(data_file)

c = {}
for a in data:
  c.setdefault(a["idFactura"],[]).append(a)

for a in c:
  print '{'
  print '"idFactura": ', c[a][0]['idFactura'], ","
  print '"IdUsuario": ',c[a][0]['idUsuario'], ","
  print '"Fecha": ', '"',c[a][0]['fecha'],'"', ","
  total = 0
  for x in c[a]:
    total = total + c[a][c[a].index(x)]['totalAPagar']
  print '"TotalAPagar": ',total, ","
  print '"estoyPagando": ['
  i = 1
  for x in c[a]:
    print "{"
    print '"idPublicacion":', c[a][c[a].index(x)]['idPublicacion'], ","
    print '"TipoSuscripcion":','"',c[a][c[a].index(x)]['nombre'], '"'
    if i == len(c[a]):
      print "}"
    else:
      print "},"
    i = i+1
  print ']'
  print '}'

\end{verbatim}

\textbf{Nota:} para la coleccion de publicaciones no tuvimos que hacer ningun algoritmo.

\newpage
\subsection{Consultas SQL}

A continuacion listamos las consultas realizadas para conseguir los datos necesarios para nuestras colecciones:
\subsubsection{SQL de Publicaciones}


\begin{verbatim}
select p.idPublicacion,   
CASE
  when s.idPublicacion is not null and (v.idPublicacion is not null or su.idPublicacion is not null) then "mixto"
  when s.idPublicacion is not null then "servicio"
    when v.idPublicacion is not null then "articulo"
    when su.idPublicacion is not null then "articulo"
END
as TipoPublicacion
from publicacion p
left join servicio s on s.idPublicacion = p.idPublicacion
left join venta v on v.idPublicacion = p.idPublicacion
left join subasta su on su.idPublicacion = p.idPublicacion
\end{verbatim}


\subsubsection{SQL de Compras}

\begin{verbatim}
select c.idCompra, c.idUsuario as idUsarioComprador,p.idUsuario as idUsuarioVendedor,c.fecha, c.cantidad, c.idPublicacion, p.precio, tp.costo, tp.porcentajeVenta, tp.nombre, ca.puntaje
from compra c 
join publicacion p
 
on c.idPublicacion = p.idPublicacion
join tipo_de_publicacion tp
on p.idTipoDePublicacion = tp.idTipoPublicacion
join calificacion ca 
on c.idCompra = ca.idCompra

\end{verbatim}

\subsubsection{SQL de Facturas}
\begin{verbatim}
select f.idFactura, p.idUsuario, f.fecha, f.totalAPagar, p.idPublicacion, tp.nombre

from factura f

inner join corresponde c
on c.idFactura = f.idFactura

inner join publicacion p
on c.idPublicacion = p.idPublicacion

inner join tipo_de_publicacion tp
on tp.idTipoPublicacion = p.idTipoDePublicacion

\end{verbatim}
\newpage
\subsection{Implementacion MongoDB}
Para implementar la base de datos de documentos en mongoDB utilizamos los JSON de las colecciones descriptos anteriormente y los importamos de la siguiente forma:

\begin{verbatim}
mongoimport --db tp2 --collection compras --drop --file ./compra_mongo.json 
mongoimport --db tp2 --collection facturas --drop --file ./factura_mongo.json
mongoimport --db tp2 --collection publicacion --drop --file ./publicacion_mongo.json
\end{verbatim}
%%%%%%%%%%%%%%%%%%%%%%%%%%%%%%%%%%%%%%%%%%%%%%%%%%%%%%%%%%%%%%%%%%%%%%%%%%%%%%%
%% Parte 2                                                                   %%
%%%%%%%%%%%%%%%%%%%%%%%%%%%%%%%%%%%%%%%%%%%%%%%%%%%%%%%%%%%%%%%%%%%%%%%%%%%%%%%
\section{Parte 2 - Map Reduce}
\subsection{Ejercicio 1}
El importe total de ventas por usuario.


\begin{lstlisting}
var ej1_m = function(){ 
  emit(this.Publicacion.idUsuario, this.Publicacion.Precio * this.cantidad) 
}

var ej1_r = function(k, vs){
  return Array.sum(vs);
}

\end{lstlisting}


\subsection{Ejercicio 2}
La reputaci\'on hist\'orica de cada usuario seg\'un la calificaci\'on.

\begin{lstlisting}
var ej2_m = function(){
  if(this.calificacionVendedor != null) emit(this.idUsuario, this.calificacionVendedor);
  if(this.calificacionComprador != null) emit(this.Publicacion.idUsuario, this.calificacionComprador);
}

var ej2_r = function(k, vs){
  return ( Array.sum(vs) / vs.length);
}
\end{lstlisting}

\subsection{Ejercicio 3}
Las operaciones con comisi\'on m\'as alta.

\begin{lstlisting}
var ej3_m = function(){
  var getComision = function(compra){
   var subs = compra.Publicacion;
    return (subs.Precio * compra.cantidad) * (subs.PorcentajeVenta / 100);
  };

  emit("todos", {
  'idCompra': this.idCompra,
  'comision': getComision(this)});
};

var ej3_r = function(k, vs){
  var max = Math.max(...vs.map(function(v){
      return v.comision;
  }));
  return {"resultado": vs.filter(function(v){
    return v.comision >= max;
  }), "max": max};
};

\end{lstlisting}



\subsection{Ejercicio 4}
El monto total facturado por año.

\begin{lstlisting}
var ej4_m = function(){
  var getAno = function(fecha){
    return fecha.trim().substring(0,4);
  }
  emit(getAno(this.Fecha), this.TotalAPagar)  
}

var ej4_r = function(k, vs){
  return Array.sum(vs);
}

\end{lstlisting}

\subsection{Ejercicio 5}
El monto total facturado por año de las operaciones pertenecientes a usuarios con suscripciones Rubi Oriente.

\begin{lstlisting}
var ej5_m1 = function(){

  var esRuby = function(e, i, arr){
    return e.TipoSuscripcion.trim() == "RubiDeOriente"
  };
  
  var getAno = function(fecha){
    return fecha.trim().substring(0,4);
  }

  emit(
    { 
      "ano": getAno(this.Fecha),
      "usuario": this.IdUsuario
    }, 
    {
      "totalAPagar": this.TotalAPagar,
      "esRuby": this.estoyPagando.some(esRuby)
    }); 
};

var ej5_r1 = function(k, vs){
  if (vs.some(function(a){
    return a.esRuby;  
  }))
  {
    return Array.sum(vs.map(function(e){return e.totalAPagar}));  
  }
}


var ej5_m2 = function(){
  if (this.value.esRuby)
  {
    emit(this._id.ano, this.value.totalAPagar);
  }
}

var ej5_r2 = function(k, vs){
  return Array.sum(vs);
}
\end{lstlisting}


\subsection{Ejercicio 6}
El total de publicaciones por tipo de publicaci\'on (productos, servicios o mixtas).

\begin{lstlisting}[caption=Ejercicio 6]
var ej6_m = function(){
  emit(this.TipoPublicacion, 1);
}

var ej6_r = function(k,vs){
  return  vs.length;
};
\end{lstlisting}
%%%%%%%%%%%%%%%%%%%%%%%%%%%%%%%%%%%%%%%%%%%%%%%%%%%%%%%%%%%%%%%%%%%%%%%%%%%%%%%
%% Parte 3                                                                   %%
%%%%%%%%%%%%%%%%%%%%%%%%%%%%%%%%%%%%%%%%%%%%%%%%%%%%%%%%%%%%%%%%%%%%%%%%%%%%%%%
\section{Parte 3 - Sharding}

\subsection{Inserccion de datos en la base de datos}
Para insertar 500000 registros en la base de datos lo que hicimos fue generar un script en Python que nos devuelva 500000 publicaciones. 
Nota: A la coleccion de publicaciones le agregamos un atributo extra "idUsuarioVendedor" para poder usarlo como atributo clave a la hora de ahcer experimentos de sharding.

El pseudocodigo para la generacion de los 500000 registros:
\begin{verbatim}
import json
from pprint import pprint
import random
from random import randint


foo = ['articulo', 'servicio', 'mixto']
print 'var pubs = ['
for x in range(1,500000):
  print '{'
  print '"idPublicacion":', x,","
  print '"TipoPublicacion":', ",",random.choice(foo),","
  print '"idUsuarioVendedor":',randint(0,8000)
  print '},'
print ']'

\end{verbatim}


A continuacion, para insertar estos datos en la base de datos de mongo, hicimos un scrpit de JavaScript para correrlo desde el Shell de mongo y que pueda importar todos estos datos en los distintos shards

\section{Indice Simple}

\subsection{Resultados}

poner que devuelve db.publicaciones.getShardDistribution() y sh.status() 

\subsection{Creacion de Shards}

Poner como hicimos para crear los shards

\subsection{Evolucion de Shards}
Poner el grafico y explicarlo

\section{Indice Hash}

\subsection{Resultados}

\emph{poner que devuelve \color{red}db.publicaciones.getShardDistribution() y sh.status() } 

\subsection{Creacion de Shards}

Poner como hicimos para crear los shards

\subsection{Evolucion de Shards}
Poner el grafico y explicarlo


\section{Indice Simple vs Indice Hash}
aca escribir la conclusion 
%%%%%%%%%%%%%%%%%%%%%%%%%%%%%%%%%%%%%%%%%%%%%%%%%%%%%%%%%%%%%%%%%%%%%%%%%%%%%%%
%% Conclusiones                                                              %%
%%%%%%%%%%%%%%%%%%%%%%%%%%%%%%%%%%%%%%%%%%%%%%%%%%%%%%%%%%%%%%%%%%%%%%%%%%%%%%%

\end{document}
